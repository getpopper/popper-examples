\documentclass{article}[12pt]

\usepackage{hyperref}
\hypersetup{
    colorlinks=true,
    linkcolor=blue,
    filecolor=magenta,      
    urlcolor=cyan,
}

\usepackage{graphicx}
\graphicspath{ {./../output/figures/} }

\title{Flu Shot Learning: Prediction H1N1 and Seasonal Flu Vaccines}
\author{Anders Poirel \\ University of California, Santa Cruz}
\date{\today}

\begin{document}
    
\maketitle

\section{Introduction}

This paper describes our approach in 
the \href{https://www.drivendata.org/competitions/66/flu-shot-learning/}{Flu Shot Learning}
on Driven Data \cite{driven-data}
The goal of the competition is to predict how likely individuals are to receive their H1N1 and
seasonal flue vaccines. Specifically, participants are asked to predict a probability for
each vaccine. Competition ranking is based on the ROC AUC of predictions on a hold-out
test set.

\section{Model}

Our approach uses a logistic regression model. Our pipeline was developed using
\verb|pandas| \cite{scikit-learn} and \verb|scikit-learn| \cite{scikit-learn}. 
Numerical features were standardized and imputed using the K-nearest neighbors
algorithm. Categorical features were one-hot encoded and imputed with ``missing''
flags. \\
We tuned a single hyper-parameter for this pipeline, logistic regression's regularization
parameter $C$, using cross-validation on 5 folds. 

\section{Results}

\subsection{Cross-validation}

Cross-validation scores suggest that the model does not over-fit for any choice
of $C$. Indeed, performance degrades for low values of $C$ (stronger regularization).
The best tested value is $C = 0.464$, although selecting a higher value does not 
appear to affect performance significantly.

\begin{figure}
\caption{Cross validation mean scores}
\includegraphics[width=\textwidth]{reg_performance.png}
\end{figure}

\subsection{Hidden test set}

After submission to the competition website, the model's predictions scored 0.8342 ROC AUC on the hidden test set,
enough to beat the organizer's benchmark (.8185). At time of writing, this score places
181st out of 948 on the competition leaderboard. 

\section{Future Work}

Our  model's validation curve shows few signs of over-fitting. As such, it is likely that
higher scores can be achieved by using a more flexible model (e.g. gradient boosted trees).
Furthermore, model stacking will generally improve results \cite{elements}, in particular
in this type of machine learning competitions where test sets are guaranteed to be sourced
from the same distribution as training data.

\bibliography{references}{}
\bibliographystyle{plain}

\end{document}